\subsubsection{Observations file}
This file is used to specify the inducing field parameters, anomaly type, observation locations, and the observed magnetic anomalies with estimated standard deviation. The values of parameters specifying the inducing field anomaly type and observation locations are identical to those in \fileName{locations} file. The output of the forward modelling program \codeName{MAGFOR3D} has the same structure except that the column of standard deviations for the error is omitted. The following is the file structure of the observations file:
\begin{fileExample}
\begin{tabular}{|ccccccc|}
\hline
! & comment & & & & & \\
Incl & Decl & geomag & & & & \\
AIncl & Adecl & idir & & & & \\
ndat & & & & & & \\
E$_1$ & N$_1$ & ELEV$_1$ & [aincl$_1$ & adecl$_1$] & Mag$_1$ & Err$_1$ \\
E$_2$ & N$_2$ & ELEV$_2$ & [aincl$2$ & adecl$_2$] & Mag$_2$ & Err$_2$ \\
\vdots & \vdots & \vdots & \vdots & \vdots & \vdots & \vdots \\
E$_{ndat}$ & N$_{ndat}$ & ELEV$_{ndat}$ & [aincl$_{ndat}$ & adecl$_{ndat}$] & Mag$_{ndat}$ & Err$_{ndat}$ \\
\hline
\end{tabular}
\end{fileExample}
%
Parameter definitions:
%
\begin{itemize}
\item[\codeName{!}] Lines starting with ! are comments.
\item[\codeName{incl/decl}] Inclination and declination of the inducing magnetic field. The declination is specified positive east with respect to the northing used in the mesh and locations files.
\item[\codeName{geomag}] Strength of the inducing field in nanoTesla (nT).
\item[\codeName{aincl/adecl/idir}] Inclination and declination of the anomaly projection. Set \codeName{idir=0} for a multi-component dataset and \codeName{idir=1} for a single component dataset where all the observations have the same inclination and declination of the anomaly projection.
\item[\codeName{ndat}] Number of observations. When single component data are specified the number of observations is equal to the number of data locations. When multi-component data are specified the number of observations will exceed the number of data locations. For example, if three-component data are specified at $N$ locations, the number of observations is $3N$.
\item[\codeName{E$_n$,N$_n$,Elev$_n$}] Easting, northing and elevation of the observation, measured in meters. Elevation should be above the topography for surface data, and below the topography for borehole data. The observation locations can be listed in any order.
\item[\codeName{ainc$_n$/adec$_n$}] Inclination and declination of the anomaly projection for observation n. This is used only when \codeName{idir = 0}. The brackets ``$[\cdots]$'' indicate that these two field are optional depending on the value of \codeName{idir}.
\item[\codeName{Mag$_n$}] Magnetic anomaly data, measured in nT.
\item[\codeName{Err$_n$}] Standard deviation of Mag$_n$. This represents the absolute error. It CANNOT be zero or negative.
\end{itemize}

\textbf{NOTE:} It should be noted that the data are \textbf{extracted anomalies} which are derived by removing the regional from the field measurements. Furthermore, the inversion program assumes that the anomalies are produced by a positive susceptibility (contrast) distribution. It is crucial that the data be prepared as such.

\subsubsection*{Examples of obs.mag file}
The following are two examples of data files. The first example file specifies a set of total field anomaly data, and the second example file provides a set of multi-component borehole data.

Example 1: single-component data
\begin{fileExample}
\begin{tabular}{|cccccl|}
\hline
! & & & surface data & & \\
45.0 & 45.0 & 50000 & 
!! Incl  & Decl & geomag\\
45.0 & 45.0 & 0 & !! AIncl & Adecl & direction of anomaly (idir)\\
144  & & & & & !! Number of data \\
0.00 & 0.00 & 1.00 & 0.174732E+02 & 0.598678E+01 & \\
0.00 & 50.00 & 1.00 & 0.265550E+02  & 0.613890E+01 & \\
0.00 & 100.00 & 1.00 & 0.311366E+02  & 0.629117E+01 & \\
\vdots & \vdots & \vdots & \vdots & \vdots & \\
1000.00 & 900.00 & 1.00 & -0.109595E+01 & 0.530682E+01 & \\
1000.00 & 950.00 & 1.00 & -0.902209E+01 & 0.523738E+01 & \\
1000.00 & 1000.00 & 1.00 & -0.397501E+01  & 0.518496E+01 & \\
\hline
\end{tabular}
\end{fileExample}

Example 2: Multi-component data
\begin{fileExample}
\begin{tabular}{|cccccl|}
\hline
! & & & & &  borehole data \\
45.0 & 45.0 & 50000 & !! Incl & Decl & geomag \\
45.0 & 45.0 & 0 & !! AIncl & Adecl &  direction of anomaly (idir) \\
144 & & & & &  !! Number of data \\
-12.50 & -137.50 & -12.50 & 0.0 & 0.0 & \begin{tabular}{cc} 0.134759E+03 & 0.200000E+01 \end{tabular} \\
-12.50 & -137.50 & -37.50 & 0.0 & 0.0 & \begin{tabular}{cc} 0.162606E+03 & 0.200000E+01  \end{tabular}\\
-12.50 & -137.50 & -62.50 & 0.0 & 0.0 & \begin{tabular}{cc} 0.165957E+03 & 0.200000E+01  \end{tabular}\\
\vdots & \vdots & \vdots & \vdots & \vdots & \\
-237.50 & -12.50 & -337.50 & 90.0 & 0.0 & \begin{tabular}{cc} 0.662445E+02 & 0.200000E+01  \end{tabular} \\
-237.50 & -12.50 & -362.50 & 90.0 & 0.0 & \begin{tabular}{cc}0.693134E+02 & 0.200000E+01  \end{tabular}\\
-237.50 & -12.50 & -387.50 & 90.0 & 0.0 & \begin{tabular}{cc}0.608605E+02 & 0.200000E+01  \end{tabular}\\
\hline
\end{tabular}
\end{fileExample}